\documentclass[a4paper,12pt]{article}
\usepackage[super,sort&compress]{natbib}
\begin{document}
\citeindextrue
\title{Final report on scripting and data management for CTA399 pulsar observations}
\author{Natalie Price-Jones}
\date{15 August, 2013}

\begin{abstract}
Pulsars, periodic radio sources, emit radiation in characteristic pulses that vary in shape and frequency. The mechanism whereby such pulses are generated is still not well understood, but with greater resolving power it might prove possible to probe the nature of the pulses in more detail. This may lead to the discovery of the intricacies of the magnetic field that produces such striking lighthouse emission. To achieve the requisite level of resolving power, this project aims to use Very Long Baseline Interferometry (VLBI) to resolve the image of the pulsar that has been scattered by interstellar medium. Such scattering provides an opportunity for interferometry with a baseline on the order of an astronomical unit, thus resolving the pulsar to picoarcsecond precision. To this end, raw voltage observations were taken at 325MHz and 150MHz on bright calibrator sources and fainter millisecond pulsars at the ARO, GMRT, LOFAR and Effelsberg telescopes.  In order to constructively add the signal from the different locations, it was necessary to write a routine to calculate the difference in the arrival time of the pulse between each of the VLBI sites and each of GMRT’s thirty antennas. In addition, detailed information was needed regarding the rapidly changing period of the millisecond pulsars so that the signal could be folded in time, increasing the signal to noise ratio. Having done this folding, we have successfully spotted both bright and millisecond pulsars at ARO and GMRT. At the moment, the project is still in the early stage of interpreting observations, confirming pulsar sightings at each individual telescope, and further observations will be taken later in the year in order to attempt the interstellar interferometry that is the primary aim. 
\end{abstract}

\section{Introduction}
\subsection{}



\begin{thebibliography}{100}

\end{thebibliography}
\end{document}